\documentclass[a4paper,12pt]{article}
\usepackage[english]{babel}
\usepackage[utf8]{inputenc}
\usepackage{graphicx}
\usepackage{titlesec}
\usepackage{caption}
\usepackage{subcaption}
\usepackage{adjustbox}
\usepackage{placeins}
\usepackage{longtable}
\titleformat*{\section}{\large\bfseries}
\titleformat*{\subsection}{\normalsize\bfseries}
\titleformat*{\subsubsection}{\vspace{-0.3cm}\normalsize\bfseries}
\title{Statistical Evaluation Report}
\date{\vspace{-10ex}}
\begin{document}
\maketitle
\section{Evaluation Overview}The system performed a 10-fold cross validation for the 3 models in Tbl \ref{tbl:models}. The models were compared against the first baseline model. 

\begin{table}[h!]
\centering
\adjustbox{max width=\linewidth}{
\begin{tabular}{|l|l|p{11cm}|} \hline
Index & Algorithm& Feature Set\\ \hline
M0 & LibLinear & nGrams: 2 \\ \hline
M1 & LibLinear & nGrams: 2, nGrams: 3 \\ \hline
M2 & LibLinear & nGrams: 2, nGrams: 4 \\ \hline
\end{tabular}
}\caption{Evaluated models with classifier algorithm and feature sets}
\label{tbl:models}
\end{table}

The models were evaluated on the dataset Boston. Their performance was assessed with the Weighted F-Measure. In the analysis, the models thus represent levels of the independent variable, while the performance measures are dependent variables.
\FloatBarrier
\section{Results}
Throughout the report, p-values are annotated if they are significant. While {\footnotesize *} indicates low significance ($p<\alpha=0.10$), the annotations {\footnotesize **} and {\footnotesize ***} represent medium ($p<\alpha=0.05$) and high significance ($p<\alpha=0.01$).\FloatBarrier
\subsection{Weighted F-Measure}
The Weighted F-Measure samples drawn from the 10-fold cross validation and the 3 models are presented in Tbl. \ref{tbl:WeightedF-Measure}.
See Fig. \ref{fig:boxPlotWeightedF-Measure} for a Box-Whisker plot of these samples. 
\begin{table}[h!]
\centering
\adjustbox{max width=\linewidth}{
\begin{tabular}{|l|l|l|l|l|l|l|l|l|l|l|l|} \hline
Classifier & \multicolumn{10}{|c|}{Weighted F-Measure per fold  [\%]}& Average\\ \hline
 & 1 & 2 & 3 & 4 & 5 & 6 & 7 & 8 & 9 & 10 &  \\ \hline
M0 & 95.53 & 95.30 & 94.43 & 95.48 & 94.32 & 95.53 & 95.35 & 94.43 & 95.39 & 94.25 & 95.00 \\ \hline
M1 & 95.24 & 94.50 & 95.50 & 95.44 & 95.39 & 95.50 & 94.50 & 95.31 & 94.41 & 94.32 & 95.01 \\ \hline
M2 & 96.00 & 97.50 & 96.00 & 94.50 & 97.00 & 95.10 & 97.00 & 98.20 & 95.30 & 97.30 & 96.39 \\ \hline
\end{tabular}
}\caption{Samples of the Weighted F-Measure drawn from the 10-fold cross validation and the 3 models}
\label{tbl:WeightedF-Measure}
\end{table}

\subsubsection{Parametric Testing}The system compared the 3 models using the \emph{repeated-measures one-way ANOVA}. Mauchly's test indicated a weak violation of Sphericity ($p_{HF}=0.010, p_{GG}=0.013, \sigma^2=0.000, p=0.094, \alpha=0.10$). Given that the assumptions are violated, the following test may be corrupted. The repeated-measures one-way ANOVA showed strong significant differences between the performances of the models ($df=2.000, F=7.343, p=0.005, \alpha=0.01$).\\ 

 The system performed the \emph{Dunett's test} post-hoc. Given that the assumptions are violated, the following test may be corrupted. The Dunett's test partly showed strong significant differences between the performances of the models ($\alpha=0.01$, Tbl. \ref{tbl:Dunett'stestWeightedF-Measure}). These results do not allow for a strict ordering of all models. The ordering is visualized in Fig. \ref{fig:graphOrderingWeightedF-MeasureParametric}. 
\begin{table}[h!]
\centering
\adjustbox{max width=\linewidth}{
\begin{tabular}{|l|l|} \hline
 & M0\\ \hline
M1 & 1.000 \\ \hline
M2 & 0.000{\footnotesize ***} \\ \hline
\end{tabular}
}\caption{P-values from the Dunett's test for Weighted F-Measure}
\label{tbl:Dunett'stestWeightedF-Measure}
\end{table}

\subsubsection{Non-Parametric Testing}The system compared the 3 models using the \emph{Friedman test}. The Friedman test did not show significant differences between the performances of the models ($Q=4.200, p=0.122, \alpha=0.10$).\\ 

 The system performed the \emph{Pairwise Wilcoxon signed-rank test} post-hoc. The Pairwise Wilcoxon signed-rank test partly showed medium significant differences between the performances of the models for non-adjusted p-values ($\alpha=0.05$, Tbl. \ref{tbl:PairwiseWilcoxonsigned-ranktestWeightedF-Measure}). These results do not allow for a strict ordering of all models. The ordering is visualized in Fig. \ref{fig:graphOrderingWeightedF-MeasureNon-Parametric}. It partly showed weak significant differences for adjusted p-values ($\alpha=0.10$, Tbl. \ref{tbl:PairwiseWilcoxonsigned-ranktestWeightedF-MeasureAdjusted}).

 
\begin{table}[h!]
\centering
\adjustbox{max width=\linewidth}{
\begin{tabular}{|l|l|} \hline
 & M0\\ \hline
M1 & 0.922 \\ \hline
M2 & 0.037{\footnotesize **} \\ \hline
\end{tabular}
}\caption{P-values from the Pairwise Wilcoxon signed-rank test for Weighted F-Measure}
\label{tbl:PairwiseWilcoxonsigned-ranktestWeightedF-Measure}
\end{table}


\begin{table}[h!]
\centering
\begin{subfigure}[b]{4cm}
\centering
\adjustbox{max width=\linewidth}{
\begin{tabular}{|l|l|} \hline
 & M0\\ \hline
M1 & 0.922 \\ \hline
M2 & 0.074{\footnotesize *} \\ \hline
\end{tabular}
}\caption{hochberg}
\end{subfigure}
\begin{subfigure}[b]{4cm}
\centering
\adjustbox{max width=\linewidth}{
\begin{tabular}{|l|l|} \hline
 & M0\\ \hline
M1 & 0.922 \\ \hline
M2 & 0.074{\footnotesize *} \\ \hline
\end{tabular}
}\caption{holm}
\end{subfigure}
\begin{subfigure}[b]{4cm}
\centering
\adjustbox{max width=\linewidth}{
\begin{tabular}{|l|l|} \hline
 & M0\\ \hline
M1 & 1.000 \\ \hline
M2 & 0.074{\footnotesize *} \\ \hline
\end{tabular}
}\caption{bonferroni}
\end{subfigure}
\caption{Adjusted p-values from the Pairwise Wilcoxon signed-rank test for Weighted F-Measure}
\label{tbl:PairwiseWilcoxonsigned-ranktestWeightedF-MeasureAdjusted}
\end{table}

\FloatBarrier
\section{Summary}
The system performed parametric testing of the 3 models using a repeated-measures one-way ANOVA and a Dunett's test post-hoc. The tests showed strong significant differences in performance for the Weighted F-Measure. \\ 

The system performed non-parametric testing of the 3 models using a Friedman test and a Pairwise Wilcoxon signed-rank test post-hoc. The tests did not show significant differences in performance for the Weighted F-Measure. \\ 

\FloatBarrier
\section{Appendix}
\begin{figure}
\centering
\includegraphics[width=1\linewidth]{boxPlotWeightedF-Measure}
\caption{Box-Whisker-Plot of Weighted F-Measure samples. Red dots indicate means.}
\label{fig:boxPlotWeightedF-Measure}
\end{figure}

\begin{figure}
\centering
\includegraphics[width=1\linewidth]{graphOrderingWeightedF-MeasureParametric}
\caption{Directed graph of significant differences between the models and Weighted F-Measure as determined by the Parametric post-hoc test.}
\label{fig:graphOrderingWeightedF-MeasureParametric}
\end{figure}

\begin{figure}
\centering
\includegraphics[width=1\linewidth]{graphOrderingWeightedF-MeasureNon-Parametric}
\caption{Directed graph of significant differences between the models and Weighted F-Measure as determined by the Non-Parametric post-hoc test.}
\label{fig:graphOrderingWeightedF-MeasureNon-Parametric}
\end{figure}

\end{document}